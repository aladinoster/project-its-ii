\documentclass[]{book}
\usepackage{lmodern}
\usepackage{amssymb,amsmath}
\usepackage{ifxetex,ifluatex}
\usepackage{fixltx2e} % provides \textsubscript
\ifnum 0\ifxetex 1\fi\ifluatex 1\fi=0 % if pdftex
  \usepackage[T1]{fontenc}
  \usepackage[utf8]{inputenc}
\else % if luatex or xelatex
  \ifxetex
    \usepackage{mathspec}
  \else
    \usepackage{fontspec}
  \fi
  \defaultfontfeatures{Ligatures=TeX,Scale=MatchLowercase}
\fi
% use upquote if available, for straight quotes in verbatim environments
\IfFileExists{upquote.sty}{\usepackage{upquote}}{}
% use microtype if available
\IfFileExists{microtype.sty}{%
\usepackage{microtype}
\UseMicrotypeSet[protrusion]{basicmath} % disable protrusion for tt fonts
}{}
\usepackage[b5paper,tmargin=1.5cm,bmargin=1.5cm,lmargin=1.5cm,rmargin=1.5cm]{geometry}
\usepackage{hyperref}
\PassOptionsToPackage{usenames,dvipsnames}{color} % color is loaded by hyperref
\hypersetup{unicode=true,
            pdftitle={Intelligent Transportation Systems ITS - II},
            pdfauthor={Andres Ladino - Angelo Furno},
            colorlinks=true,
            linkcolor=Maroon,
            citecolor=Blue,
            urlcolor=Blue,
            breaklinks=true}
\urlstyle{same}  % don't use monospace font for urls
\usepackage{color}
\usepackage{fancyvrb}
\newcommand{\VerbBar}{|}
\newcommand{\VERB}{\Verb[commandchars=\\\{\}]}
\DefineVerbatimEnvironment{Highlighting}{Verbatim}{commandchars=\\\{\}}
% Add ',fontsize=\small' for more characters per line
\usepackage{framed}
\definecolor{shadecolor}{RGB}{248,248,248}
\newenvironment{Shaded}{\begin{snugshade}}{\end{snugshade}}
\newcommand{\AlertTok}[1]{\textcolor[rgb]{0.94,0.16,0.16}{#1}}
\newcommand{\AnnotationTok}[1]{\textcolor[rgb]{0.56,0.35,0.01}{\textbf{\textit{#1}}}}
\newcommand{\AttributeTok}[1]{\textcolor[rgb]{0.77,0.63,0.00}{#1}}
\newcommand{\BaseNTok}[1]{\textcolor[rgb]{0.00,0.00,0.81}{#1}}
\newcommand{\BuiltInTok}[1]{#1}
\newcommand{\CharTok}[1]{\textcolor[rgb]{0.31,0.60,0.02}{#1}}
\newcommand{\CommentTok}[1]{\textcolor[rgb]{0.56,0.35,0.01}{\textit{#1}}}
\newcommand{\CommentVarTok}[1]{\textcolor[rgb]{0.56,0.35,0.01}{\textbf{\textit{#1}}}}
\newcommand{\ConstantTok}[1]{\textcolor[rgb]{0.00,0.00,0.00}{#1}}
\newcommand{\ControlFlowTok}[1]{\textcolor[rgb]{0.13,0.29,0.53}{\textbf{#1}}}
\newcommand{\DataTypeTok}[1]{\textcolor[rgb]{0.13,0.29,0.53}{#1}}
\newcommand{\DecValTok}[1]{\textcolor[rgb]{0.00,0.00,0.81}{#1}}
\newcommand{\DocumentationTok}[1]{\textcolor[rgb]{0.56,0.35,0.01}{\textbf{\textit{#1}}}}
\newcommand{\ErrorTok}[1]{\textcolor[rgb]{0.64,0.00,0.00}{\textbf{#1}}}
\newcommand{\ExtensionTok}[1]{#1}
\newcommand{\FloatTok}[1]{\textcolor[rgb]{0.00,0.00,0.81}{#1}}
\newcommand{\FunctionTok}[1]{\textcolor[rgb]{0.00,0.00,0.00}{#1}}
\newcommand{\ImportTok}[1]{#1}
\newcommand{\InformationTok}[1]{\textcolor[rgb]{0.56,0.35,0.01}{\textbf{\textit{#1}}}}
\newcommand{\KeywordTok}[1]{\textcolor[rgb]{0.13,0.29,0.53}{\textbf{#1}}}
\newcommand{\NormalTok}[1]{#1}
\newcommand{\OperatorTok}[1]{\textcolor[rgb]{0.81,0.36,0.00}{\textbf{#1}}}
\newcommand{\OtherTok}[1]{\textcolor[rgb]{0.56,0.35,0.01}{#1}}
\newcommand{\PreprocessorTok}[1]{\textcolor[rgb]{0.56,0.35,0.01}{\textit{#1}}}
\newcommand{\RegionMarkerTok}[1]{#1}
\newcommand{\SpecialCharTok}[1]{\textcolor[rgb]{0.00,0.00,0.00}{#1}}
\newcommand{\SpecialStringTok}[1]{\textcolor[rgb]{0.31,0.60,0.02}{#1}}
\newcommand{\StringTok}[1]{\textcolor[rgb]{0.31,0.60,0.02}{#1}}
\newcommand{\VariableTok}[1]{\textcolor[rgb]{0.00,0.00,0.00}{#1}}
\newcommand{\VerbatimStringTok}[1]{\textcolor[rgb]{0.31,0.60,0.02}{#1}}
\newcommand{\WarningTok}[1]{\textcolor[rgb]{0.56,0.35,0.01}{\textbf{\textit{#1}}}}
\usepackage{longtable,booktabs}
\usepackage{graphicx,grffile}
\makeatletter
\def\maxwidth{\ifdim\Gin@nat@width>\linewidth\linewidth\else\Gin@nat@width\fi}
\def\maxheight{\ifdim\Gin@nat@height>\textheight\textheight\else\Gin@nat@height\fi}
\makeatother
% Scale images if necessary, so that they will not overflow the page
% margins by default, and it is still possible to overwrite the defaults
% using explicit options in \includegraphics[width, height, ...]{}
\setkeys{Gin}{width=\maxwidth,height=\maxheight,keepaspectratio}
\IfFileExists{parskip.sty}{%
\usepackage{parskip}
}{% else
\setlength{\parindent}{0pt}
\setlength{\parskip}{6pt plus 2pt minus 1pt}
}
\setlength{\emergencystretch}{3em}  % prevent overfull lines
\providecommand{\tightlist}{%
  \setlength{\itemsep}{0pt}\setlength{\parskip}{0pt}}
\setcounter{secnumdepth}{5}
% Redefines (sub)paragraphs to behave more like sections
\ifx\paragraph\undefined\else
\let\oldparagraph\paragraph
\renewcommand{\paragraph}[1]{\oldparagraph{#1}\mbox{}}
\fi
\ifx\subparagraph\undefined\else
\let\oldsubparagraph\subparagraph
\renewcommand{\subparagraph}[1]{\oldsubparagraph{#1}\mbox{}}
\fi

%%% Use protect on footnotes to avoid problems with footnotes in titles
\let\rmarkdownfootnote\footnote%
\def\footnote{\protect\rmarkdownfootnote}

%%% Change title format to be more compact
\usepackage{titling}

% Create subtitle command for use in maketitle
\newcommand{\subtitle}[1]{
  \posttitle{
    \begin{center}\large#1\end{center}
    }
}

\setlength{\droptitle}{-2em}

  \title{Intelligent Transportation Systems ITS - II}
    \pretitle{\vspace{\droptitle}\centering\huge}
  \posttitle{\par}
    \author{Andres Ladino - Angelo Furno}
    \preauthor{\centering\large\emph}
  \postauthor{\par}
      \predate{\centering\large\emph}
  \postdate{\par}
    \date{29/11/2018}

\usepackage[utf8]{inputenc} % Accent inputs
\usepackage[english]{babel} % English language/hyphenation
\usepackage[T1]{fontenc} % Font encoding
\usepackage{microtype} % Rendering
\usepackage{xspace} % Space optimization
\usepackage{fourier}
\usepackage{booktabs}
\usepackage{amsthm}
\makeatletter
\def\thm@space@setup{%
  \thm@preskip=8pt plus 2pt minus 4pt
  \thm@postskip=\thm@preskip
}
\makeatother

\usepackage{amsthm}
\newtheorem{theorem}{Theorem}[chapter]
\newtheorem{lemma}{Lemma}[chapter]
\theoremstyle{definition}
\newtheorem{definition}{Definition}[chapter]
\newtheorem{corollary}{Corollary}[chapter]
\newtheorem{proposition}{Proposition}[chapter]
\theoremstyle{definition}
\newtheorem{example}{Example}[chapter]
\theoremstyle{definition}
\newtheorem{exercise}{Exercise}[chapter]
\theoremstyle{remark}
\newtheorem*{remark}{Remark}
\newtheorem*{solution}{Solution}
\begin{document}
\maketitle

{
\hypersetup{linkcolor=black}
\setcounter{tocdepth}{1}
\tableofcontents
}
\hypertarget{its-for-smart-mobility}{%
\chapter*{ITS for Smart Mobility}\label{its-for-smart-mobility}}
\addcontentsline{toc}{chapter}{ITS for Smart Mobility}




\begin{figure}

{\centering \includegraphics{images/01-car} 

}

\caption{New connected vehicles
\href{Taken\%20from:\%20https://pixabay.com}{https://pixabay.com}.}\label{fig:car}
\end{figure}

\hypertarget{context}{%
\section*{Context}\label{context}}
\addcontentsline{toc}{section}{Context}

Traffic congestion on urban roads is a problem of great interest
nowadays since it strongly affects security and pollution. Workfoce
centralization, population and economic growth alongside with continuous
urbanization are the main causes of traffic congestion. As cities strive
to update/expand the current infrastructures the development of
Information Technologies bring new possiblities as an alternative
solution for transportation systems.

The current project aims to explore some of the new technologies used in
the so called Intelligent Transportation Systems (ITS). They objective
is to study to a certain level of detail some of the new traffic
management systems that will conduct new ways of transportation in the
XXI century. The general idea is based on the fact that information
collected by sensors within traffic networks or in-vehicles sensors can
collect information regarding the traffic condition, perform estimation
of unknown traffic states and decide on specific actions to modify this
state.

\begin{center}\rule{0.5\linewidth}{\linethickness}\end{center}

\hypertarget{projects}{%
\section*{Projects}\label{projects}}
\addcontentsline{toc}{section}{Projects}

Specially, in this case the project will be focused in four main
activities:

\begin{itemize}
\tightlist
\item
  \textbf{Trafic signal design:} This projects aims to study how to
  optimally deal with the design of light cycles to optimize the traffic
  performance. Based on information collected by infrastructure based
  sensors the traffic state can be determined and specific decisions in
  terms on green/red light times can be dynamically adapted.
\item
  \textbf{Vehicle Platooning:} New in-vechicle sensors create situations
  in which vehicles may exchange information to improve traffic
  conditions. This project aims to study control algorithms implemented
  in the V2V (Vehicle-vehicle) communication layer that can be used to
  design traffic decisions.
\item
  \textbf{Density reconstruction:} Sensors installed in traffic
  infrastructures provide information regarding the traffic state.
  Nevertheless the solution is not scalable to cover large cities due to
  the economical leverage required to deploy this technology. Algorithms
  to reconstruct traffic information may provide a promising future for
  accurately determine the traffic state of a city. The aim of this
  project is to study how multiple sources of information can be
  integrated to reconstruct traffic variables within a city.
\end{itemize}

\begin{center}\rule{0.5\linewidth}{\linethickness}\end{center}

\hypertarget{general-objectives}{%
\section*{General Objectives}\label{general-objectives}}
\addcontentsline{toc}{section}{General Objectives}

\begin{itemize}
\tightlist
\item
  Identify new technologies implemented in the Intelligent Transporation
  Systems and investigate how these technologies are deployed in real
  systems.
\item
  Define and determine adequate models that are suitable for deploying
  new ITS technologies.
\item
  Develop specific solutions for ITS that can be tested under
  pre-defined scenarios.
\end{itemize}

\hypertarget{project-information}{%
\chapter*{Project information}\label{project-information}}
\addcontentsline{toc}{chapter}{Project information}

\hypertarget{deliverables}{%
\section*{Deliverables}\label{deliverables}}
\addcontentsline{toc}{section}{Deliverables}

The project is mainly composed in 3 phases.

\begin{enumerate}
\def\labelenumi{\arabic{enumi}.}
\item
  \textbf{Problem identification \& literature review:} This phases
  consists in:

  \begin{enumerate}
  \def\labelenumii{\arabic{enumii}.}
  \item
    Identify particularly the problem under study, meaning the system to
    be controlled and the way it is assesed.
  \item
    Retrieve bibliographical information about the problem under study.
  \item
    Summarize the already proposed alternatives in the existing
    literature.
  \end{enumerate}

  The main objective of this phase is to understand what are the main
  challenges when solving the specific project under study and to
  present in general ways how this problem has been solved.
\item
  \textbf{Setting up a suitable model:} Once the problem has been
  identified the main objective is to precisely describe the traffic
  models that are suitable for the approach. For this phase the stages
  are divided as:

  \begin{enumerate}
  \def\labelenumii{\arabic{enumii}.}
  \tightlist
  \item
    Determine a specific traffic model that can be used for the
    corresponding situation
  \item
    Define the scenario in which the solution should be tested
  \item
    Pre establish parameters and requirements for the solution to be
    implemented.
  \end{enumerate}
\item
  \textbf{Experimental results:} Finally, the main objective is to
  perform a validation and solution for the problem under study. Several
  tools are provided for this purpose like micro/macro
\end{enumerate}

\begin{center}\rule{0.5\linewidth}{\linethickness}\end{center}

\hypertarget{reporting}{%
\section*{Reporting}\label{reporting}}
\addcontentsline{toc}{section}{Reporting}

In order to fullfill the requirements for each phase each group should
provide a report as follows:

\begin{itemize}
\item
  \emph{Report 1:} Summarizes the results of the 1. Problem
  identification \& literature review and 2. Setting up a suitable
  model. (\textbf{\emph{Due date: January 9th, 2018}})
\item
  \emph{Report 2:} Summarizes the results of the phase 3.Experimental
  results (\textbf{\emph{Due date: January 23rd, 2019}})
\end{itemize}

\hypertarget{project-1-signalized-traffic}{%
\chapter{Project 1: Signalized
Traffic}\label{project-1-signalized-traffic}}

The main objective of this project is to design \emph{traffic light
signal} controls in order to optimize particular traffic conditions.
Traffic signals regularly pre-establish fixed values \emph{red} or
\emph{green} for a particular intersection. In fact the behavior can be
modeled as:



\begin{figure}

{\centering \includegraphics{its-2-project_files/figure-latex/tlight-1} 

}

\caption{Traffic light signal example.}\label{fig:tlight}
\end{figure}

The relationship between the switched green/red time in a traffic light
can be represented by a pulse signal (Figure \ref{fig:tlight}). The red
line in the figure represents the \emph{duty cycle} which represents the
fraction of time the light was in green (\(1\)) with respect to the
total cycle time (\(60s\)). In this case, the main objective is to study
traffic models that can model signalized intersections and design
control laws.

\begin{center}\rule{0.5\linewidth}{\linethickness}\end{center}

\hypertarget{objectives}{%
\section*{Objectives}\label{objectives}}
\addcontentsline{toc}{section}{Objectives}

The main objective of this project is to:

\begin{enumerate}
\def\labelenumi{\arabic{enumi}.}
\tightlist
\item
  Study the fundamental aspects of traffic signal control strategies.
\item
  Obtain and simulate a macroscopic traffic model for a urban network
  with traffic signals
\item
  Create and design control strategies applied via traffic signals in
  urban traffic networks.
\item
  Compare the behavior of fixed-time traffic signal polices and
  dynamic-time traffic signal polices.
\end{enumerate}

\hypertarget{description}{%
\section*{Description}\label{description}}
\addcontentsline{toc}{section}{Description}

\hypertarget{task-1-modeling}{%
\subsection*{Task 1: Modeling}\label{task-1-modeling}}
\addcontentsline{toc}{subsection}{Task 1: Modeling}

Check models for macroscopic networks: (Grandinetti, Canudas-de-wit, and
Garin \protect\hyperlink{ref-Grandinetti2015}{2015}), (Grandinetti,
Garin, and Canudas-de-wit
\protect\hyperlink{ref-Grandinetti2016}{2015}),(Varaiya
\protect\hyperlink{ref-Varaiya2013:TR-C}{2013}). Determine the
parameters required to model a road traffic network.

\hypertarget{context-1}{%
\subsubsection*{Context}\label{context-1}}
\addcontentsline{toc}{subsubsection}{Context}

Before implementing a real scenario in this phase we aim to describe the
context of a virtual example. Consider the following traffic urban
corridor which obeys a simplified version of the real arterial scenario.



\begin{figure}

{\centering \includegraphics{images/p1-01-network} 

}

\caption{Example of traffic network.}\label{fig:city}
\end{figure}

The network of Figure \ref{fig:city} represents a regular corridor in a
city like. The priority for this type of corridors is to maximize the
priority of green time so the traffic does not get congested along the
network. It is important to highlight that for the proposed network

\hypertarget{questions}{%
\subsubsection*{Questions}\label{questions}}
\addcontentsline{toc}{subsubsection}{Questions}

\begin{itemize}
\tightlist
\item
  Based on a literature review, determine current existing traffic
  models for signalized traffic networks.
\item
  For the network in Figure \ref{fig:city}, build a macroscopic
  signalized traffic model.
\item
  Determine the \emph{signalized/averaged} traffic model for the case
  where lights are installed along each intersection of the corridor.
  For those roads in which direction is not explicitly defined, consider
  the direction of a regular intersection.
\end{itemize}

\hypertarget{expected-outcomes}{%
\subsubsection*{Expected outcomes}\label{expected-outcomes}}
\addcontentsline{toc}{subsubsection}{Expected outcomes}

\begin{itemize}
\item
  Present a brief summary of the existing current models for signalized
  traffic networks. At same time highlight the key features of these
  models and the remaining difficulties. Consider including references
  for the presented models.
\item
  Based on (Grandinetti, Canudas-de-wit, and Garin
  \protect\hyperlink{ref-Grandinetti2015}{2015}) obtain a model for the
  signalized network of Figure \ref{fig:city} and the parameters
  required model this corridor.
\item
  Define the set of parameters for the traffic network, notably those
  related to the fundamental diagram as well as the traffic light signal
  timings. Consider a fixed setup of this parameters for the moment and
  overall explain the reason of the choice.
\end{itemize}

\begin{center}\rule{0.5\linewidth}{\linethickness}\end{center}

\hypertarget{task-2-simulation}{%
\subsection*{Task 2: Simulation}\label{task-2-simulation}}
\addcontentsline{toc}{subsection}{Task 2: Simulation}

Dynamic simulation of open loop traffic networks. For this task be sure
to review already implemented simulations available at
\href{https://github.com/andres-ladino-ifsttar/traffic-macrosimulator}{Link}.
Get familiar with the code here developed before you enter into details
of implementation.

\hypertarget{context-2}{%
\subsubsection*{Context}\label{context-2}}
\addcontentsline{toc}{subsubsection}{Context}

Based on the model established in the \textbf{Task 1} and considering a
family of specific parameters, perform a simulation for a constant
demand value of demand in \(veh/hr\) as specified in the
Figure\ref{fig:flow}.



\begin{figure}

{\centering \includegraphics{images/p1-02-flows} 

}

\caption{Demand profiles for a specified scenario.}\label{fig:flow}
\end{figure}

\hypertarget{questions-1}{%
\subsubsection*{Questions}\label{questions-1}}
\addcontentsline{toc}{subsubsection}{Questions}

\begin{itemize}
\item
  Simulate and obtain density plots of the dynamic the behaviour when
  the \emph{duty cyle} of lights is similar for all the traffic lights.
  \(50\%\)
\item
  Does the current values create a congestion in the proposed network?
\item
  How can the performance of the network be improved? Create a second
  set of parameters for the traffic lights \emph{duty cycle}
\item
  For the existing model and according to (Grandinetti, Canudas-de-wit,
  and Garin \protect\hyperlink{ref-Grandinetti2015}{2015}) obtain the
  two different \emph{density} and \emph{flow} dynamic profiles between
  a \emph{switched} signalized traffic model and an \emph{averaged}
  traffic model
\end{itemize}

\hypertarget{expected-outcomes-1}{%
\subsubsection*{Expected outcomes}\label{expected-outcomes-1}}
\addcontentsline{toc}{subsubsection}{Expected outcomes}

\begin{itemize}
\item
  Present the dynamic profiles for density on each one of the roads
  under different setups of traffic lights. The dynamic profiles should
  specify density per road in time
\item
  Provide comparisons between different traffic signalized models and an
  analysis on how this error evolves according to different values of
  demand.
\item
  Specify a second set of parameters for \emph{duty cycle} that could
  improve the performance of the network with respect to the \(50\%\)
  case. Justify the reason of your choice and verify in results the
  desired behavior.
\end{itemize}

\begin{center}\rule{0.5\linewidth}{\linethickness}\end{center}

\hypertarget{task-3-control-strategy}{%
\subsection*{Task 3: Control strategy}\label{task-3-control-strategy}}
\addcontentsline{toc}{subsection}{Task 3: Control strategy}

For this task we aim to consider the \emph{duty cycle} as a control
input variable to regulate the flow within the traffic network. In this
case take into account that the decision variable needs somehow to be
determined dynamically via a control algorithm.

\hypertarget{context-3}{%
\subsubsection*{Context}\label{context-3}}
\addcontentsline{toc}{subsubsection}{Context}

For the Figure \ref{fig:tlight} the average value can be found as:

\begin{equation}
\bar{u} = \frac{1}{T}\int_0^T u(\tau) d\tau \label{eq:avgu}
\end{equation}

As it is illustrated in (Grandinetti, Garin, and Canudas-de-wit
\protect\hyperlink{ref-Grandinetti2016}{2015}), in Eq. \eqref{eq:avgu} is
easier to design a value for \(\bar{u}\) rather than \(u\) due to its
binary character.

\hypertarget{questions-2}{%
\subsubsection*{Questions}\label{questions-2}}
\addcontentsline{toc}{subsubsection}{Questions}

\begin{itemize}
\item
  Explain and construct a block diagram illustrating the control of the
  road traffic network via traffic signals
\item
  Design and present an algorithm that takes decisions for the traffic
  signals based on the state of the network. How would you construct the
  value of \(u\) based on the congestion state of the network.
\item
  Perform simulations of the system under a preestablished manual
  control and compare it to a situation where the control depends on the
  congestion state of the network.
\end{itemize}

\hypertarget{expected-outcomes-2}{%
\subsubsection*{Expected outcomes}\label{expected-outcomes-2}}
\addcontentsline{toc}{subsubsection}{Expected outcomes}

\begin{itemize}
\item
  Present a comparison between the traffic network with signalized
  control and without signalized control.
\item
  Apply the controller over different versions of the model notably
  \emph{averaged} and \emph{switched}
\item
  Compare the open loop situation with the closed loop situation and
  provide conclussions on the control performance.
\end{itemize}

\begin{center}\rule{0.5\linewidth}{\linethickness}\end{center}

\hypertarget{task-4-performance-evaluation}{%
\subsection*{Task 4: Performance
evaluation}\label{task-4-performance-evaluation}}
\addcontentsline{toc}{subsection}{Task 4: Performance evaluation}

Finally in order to compare it is important to determine indicators than
can be designed in order to compare the effects of introducing an
automated control strategy.

\hypertarget{context-4}{%
\subsubsection*{Context}\label{context-4}}
\addcontentsline{toc}{subsubsection}{Context}

Indicators for traffic networks are regularly expressed in terms of the
state of the network, for example the \emph{Service of Demand} (SoD)
measured in terms of flow:

\begin{equation}
SOD = \int_0^T \sum_{r \in R} f_r(\tau) d\tau \label{eq:sod}
\end{equation}

Or the \emph{Total Travel Distance} (TTD) measured in terms of the
density

\begin{equation}
TTD = \int_0^T \sum_{r \in R} \rho_r(\tau) d\tau \label{eq:ttd}
\end{equation}

\hypertarget{questions-3}{%
\subsubsection*{Questions}\label{questions-3}}
\addcontentsline{toc}{subsubsection}{Questions}

\begin{itemize}
\item
  What does the aforementioned indicators represent?
\item
  Measure the indicators over the traffic network with a manual setup of
  traffic signals \emph{open-loop} vs a controlled setup of traffic
  signals \emph{closed-loop}.
\end{itemize}

\hypertarget{expected-outcomes-3}{%
\subsubsection*{Expected outcomes}\label{expected-outcomes-3}}
\addcontentsline{toc}{subsubsection}{Expected outcomes}

\begin{itemize}
\item
  Report indicators for both cases and conclude about the results.
\item
  Provide recomendations on how to design \(u\) for the corridor
\end{itemize}

\hypertarget{sources}{%
\section*{Sources}\label{sources}}
\addcontentsline{toc}{section}{Sources}

For more details on how to deploy traffic simulations and traffic models
please check:

\begin{itemize}
\item
  \href{https://github.com/andres-ladino-ifsttar/traffic-macrosimulator}{Traffic
  Macrosimulator - Github}
\item
  For traffic models check (Grandinetti, Canudas-de-wit, and Garin
  \protect\hyperlink{ref-Grandinetti2015}{2015}) available
  \href{https://hal.archives-ouvertes.fr/hal-01188535}{Link} and
  (Grandinetti, Garin, and Canudas-de-wit
  \protect\hyperlink{ref-Grandinetti2016}{2015}) available at
  \href{https://hal.archives-ouvertes.fr/hal-01188811}{Link}
\end{itemize}

\hypertarget{project-2-vehicle-platooning}{%
\chapter{Project 2: Vehicle
Platooning}\label{project-2-vehicle-platooning}}

The main objective of this project is to design a controller for a
platoon of vehicles in order to optimize the traffic flow and reduce the
fuel consumption. A platoon of vehicles can be seen as:



\begin{figure}

{\centering \includegraphics{images/p2-cavs} 

}

\caption{Example of a vehicle platoon.}\label{fig:cav}
\end{figure}

The objective illustrated in \ref{fig:cav} is to control the
\emph{headway space} between two single vehicles in a formation of
multiple vehicles. This project is inspired in works presented on
(Duret, Wang, and Ladino
\protect\hyperlink{ref-Duret2019:ISTTT}{2019}\protect\hyperlink{ref-Duret2019:ISTTT}{a})
but more information about platoons can be found at (Ali, Garcia, and
Martinet \protect\hyperlink{ref-Ali2015:ITSM}{2015}).

\hypertarget{objectives-1}{%
\section*{Objectives}\label{objectives-1}}
\addcontentsline{toc}{section}{Objectives}

The main objective of this project is to:

\begin{enumerate}
\def\labelenumi{\arabic{enumi}.}
\tightlist
\item
  Study the fundamental problem of string stability.
\item
  Obtain and create a dynamic model for vehicle platoons and asociate
  its parameters with traffic theory.
\item
  Create and design a control strategy for the headway space and analyze
  its behavior.
\item
  Compare and analyze the behavior of parameter setups for variations of
  the proposed control strategy
\end{enumerate}

\hypertarget{description-1}{%
\section*{Description}\label{description-1}}
\addcontentsline{toc}{section}{Description}

\hypertarget{task-1-platoon-modeling}{%
\subsection*{Task 1: Platoon modeling}\label{task-1-platoon-modeling}}
\addcontentsline{toc}{subsection}{Task 1: Platoon modeling}

Consider the dynamical model presented in (Duret, Wang, and Ladino
\protect\hyperlink{ref-Duret2019:ISTTT}{2019}\protect\hyperlink{ref-Duret2019:ISTTT}{a}),
(Turri, Besselink, and Johansson
\protect\hyperlink{ref-Turri2017}{2017}) where the problem of truck
platooning is detailed.

\hypertarget{context-5}{%
\subsubsection*{Context}\label{context-5}}
\addcontentsline{toc}{subsubsection}{Context}

In general the platoon problem is a \emph{dynamic control problem} where
the objective is to regulate or mantain the value of a specific variable
within the system at a desired level. In order to produce this
regulation a \emph{controller} is requrired and in most of the cases
cases there is always a way to express the dynamical model in the
following form:

\begin{align}
\dot{x}(t) = f(x(t), u(t)) \label{eq:platoon}
\end{align}

where \(x(t), u(t)\) correspond to the state vector, and the control
vector of the system. In general a system describing this

\hypertarget{questions-4}{%
\subsubsection*{Questions}\label{questions-4}}
\addcontentsline{toc}{subsubsection}{Questions}

\begin{itemize}
\tightlist
\item
  What are the main goals of platoon strategies, and how they can
  improve the traffic in future?
\item
  Create a model for a platoon of 6 vehicles that considers a simple
  dynamic model of 2nd order. \textbf{Note: Consider for this case a
  model in which no drag is present within the model.}. This model can
  be written as in \eqref{eq:platoon} where the stability can be analyzed.
\item
  Which could be the traffic parameters in this kind of models?
\end{itemize}

\hypertarget{expected-outcomes-4}{%
\subsubsection*{Expected outcomes}\label{expected-outcomes-4}}
\addcontentsline{toc}{subsubsection}{Expected outcomes}

\begin{itemize}
\tightlist
\item
  Present a brief summary on the motivations to create vehicle platoons
  and the main existing models that can be developed for this purpose.
\item
  Based on the work of (Duret, Wang, and Ladino
  \protect\hyperlink{ref-Duret2019:ISTTT}{2019}\protect\hyperlink{ref-Duret2019:ISTTT}{a}),
  (Wang et al. \protect\hyperlink{ref-Meng2014b:TR-C}{2014}) determine
  writedown the state equation for a platoon model of 4 vehicles.
\item
  Determine the stability properties of the model. Stability properties
  are associated to the location of zeros and poles of the transfer
  function of the model or eigen values of the matrix
  \href{https://en.wikipedia.org/wiki/Stability_theory}{See more}
\end{itemize}

\begin{center}\rule{0.5\linewidth}{\linethickness}\end{center}

\hypertarget{task-2-open-loop-simulation}{%
\subsection*{Task 2: Open loop
simulation}\label{task-2-open-loop-simulation}}
\addcontentsline{toc}{subsection}{Task 2: Open loop simulation}

\hypertarget{context-6}{%
\subsubsection*{Context}\label{context-6}}
\addcontentsline{toc}{subsubsection}{Context}

\hypertarget{questions-5}{%
\subsubsection*{Questions}\label{questions-5}}
\addcontentsline{toc}{subsubsection}{Questions}

\hypertarget{expected-outcomes-5}{%
\subsubsection*{Expected outcomes}\label{expected-outcomes-5}}
\addcontentsline{toc}{subsubsection}{Expected outcomes}

\begin{itemize}
\tightlist
\item
  Obtain the trajectories for a platoon of 4 vehicles
\end{itemize}

\hypertarget{task-3-vehicle-platoon-control}{%
\subsection*{Task 3: Vehicle platoon
control}\label{task-3-vehicle-platoon-control}}
\addcontentsline{toc}{subsection}{Task 3: Vehicle platoon control}

\begin{center}\rule{0.5\linewidth}{\linethickness}\end{center}

\hypertarget{context-7}{%
\subsubsection*{Context}\label{context-7}}
\addcontentsline{toc}{subsubsection}{Context}

\hypertarget{questions-6}{%
\subsubsection*{Questions}\label{questions-6}}
\addcontentsline{toc}{subsubsection}{Questions}

\hypertarget{expected-outcomes-6}{%
\subsubsection*{Expected outcomes}\label{expected-outcomes-6}}
\addcontentsline{toc}{subsubsection}{Expected outcomes}

\hypertarget{task-4-performance-evaluation-1}{%
\subsection*{Task 4: Performance
evaluation}\label{task-4-performance-evaluation-1}}
\addcontentsline{toc}{subsection}{Task 4: Performance evaluation}

\hypertarget{context-8}{%
\subsubsection*{Context}\label{context-8}}
\addcontentsline{toc}{subsubsection}{Context}

\hypertarget{questions-7}{%
\subsubsection*{Questions}\label{questions-7}}
\addcontentsline{toc}{subsubsection}{Questions}

\hypertarget{expected-outcomes-7}{%
\subsubsection*{Expected outcomes}\label{expected-outcomes-7}}
\addcontentsline{toc}{subsubsection}{Expected outcomes}

\hypertarget{sources-1}{%
\section*{Sources}\label{sources-1}}
\addcontentsline{toc}{section}{Sources}

\begin{itemize}
\item
  \href{https://github.com/aladinoster/density-reconstruction}{Simulation
  results - Github}
\item
  Check (Duret, Wang, and Ladino
  \protect\hyperlink{ref-Duret2019}{2019}\protect\hyperlink{ref-Duret2019}{b})
  available \href{http://bit.ly/Hierarchical_ISTTT}{Link}
\end{itemize}

\hypertarget{project-4-density-reconstruction}{%
\chapter{Project 4: Density
Reconstruction}\label{project-4-density-reconstruction}}

\includegraphics{images/p4-01-density.png}




\begin{Shaded}
\begin{Highlighting}[]
\KeywordTok{plot}\NormalTok{(cars)  }\CommentTok{# a scatterplot}
\end{Highlighting}
\end{Shaded}

\begin{figure}
\centering
\includegraphics{its-2-project_files/figure-latex/foo-1.pdf}
\caption{\label{fig:foo}A scatterplot of the data \texttt{cars} using \textbf{base} R
graphics.}
\end{figure}

\begin{equation}
f\left(k\right)=\binom{n}{k}p^k\left(1-p\right)^{n-k} \label{eq:binom}
\end{equation}

In \eqref{eq:binom}

Below is an \texttt{align} environment \eqref{eq:align}:

\begin{Shaded}
\begin{Highlighting}[]
\KeywordTok{\textbackslash{}begin}\NormalTok{\{}\ExtensionTok{align}\NormalTok{\}}\SpecialStringTok{ }
\SpecialStringTok{g(X_\{n\}) &= g(}\SpecialCharTok{\textbackslash{}theta}\SpecialStringTok{)+g'(\{}\SpecialCharTok{\textbackslash{}tilde}\SpecialStringTok{\{}\SpecialCharTok{\textbackslash{}theta}\SpecialStringTok{\}\})(X_\{n\}-}\SpecialCharTok{\textbackslash{}theta}\SpecialStringTok{) }\SpecialCharTok{\textbackslash{}notag}\SpecialStringTok{ }\SpecialCharTok{\textbackslash{}\textbackslash{}}
\SpecialCharTok{\textbackslash{}sqrt}\SpecialStringTok{\{n\}[g(X_\{n\})-g(}\SpecialCharTok{\textbackslash{}theta}\SpecialStringTok{)] &= g'}\SpecialCharTok{\textbackslash{}left}\SpecialStringTok{(\{}\SpecialCharTok{\textbackslash{}tilde}\SpecialStringTok{\{}\SpecialCharTok{\textbackslash{}theta}\SpecialStringTok{\}\}}\SpecialCharTok{\textbackslash{}right}\SpecialStringTok{)}
\SpecialStringTok{  }\SpecialCharTok{\textbackslash{}sqrt}\SpecialStringTok{\{n\}[X_\{n\}-}\SpecialCharTok{\textbackslash{}theta}\SpecialStringTok{ ] (}\SpecialCharTok{\textbackslash{}#}\SpecialStringTok{eq:align)}
\KeywordTok{\textbackslash{}end}\NormalTok{\{}\ExtensionTok{align}\NormalTok{\} }
\end{Highlighting}
\end{Shaded}

\begin{align}
g(X_{n}) &= g(\theta)+g'({\tilde{\theta}})(X_{n}-\theta)\\
\sqrt{n}[g(X_{n})-g(\theta)] &= g'\left({\tilde{\theta}}\right)
  \sqrt{n}[X_{n}-\theta ] \label{eq:align}
\end{align}

\hypertarget{sources-2}{%
\section*{Sources}\label{sources-2}}
\addcontentsline{toc}{section}{Sources}

For more details on the project please check:

\begin{itemize}
\item
  \href{https://github.com/aladinoster/density-reconstruction}{Simulation
  results - Github}
\item
  Check (Ladino et al. \protect\hyperlink{ref-Ladino2018}{2018})
  available \href{https://hal.archives-ouvertes.fr/hal-01731356}{Link}
  and (Lovisari, Canudas-de-wit, and Kibangou
  \protect\hyperlink{ref-Lovisari2016}{2016})
  \href{https://hal.archives-ouvertes.fr/hal-01375928}{Link}
\end{itemize}

\hypertarget{reference}{%
\section*{Reference}\label{reference}}
\addcontentsline{toc}{section}{Reference}

The main reference for this project is (Ladino et al.
\protect\hyperlink{ref-Ladino2018}{2018})

\hypertarget{refs}{}
\leavevmode\hypertarget{ref-Ali2015:ITSM}{}%
Ali, Alan, Gaetan Garcia, and Philippe Martinet. 2015. ``The flatbed
platoon towing model for safe and dense platooning on highways.''
\emph{IEEE Intelligent Transportation Systems Magazine} 7 (1): 58--68.
\url{https://doi.org/10.1109/MITS.2014.2328670}.

\leavevmode\hypertarget{ref-Duret2019:ISTTT}{}%
Duret, Aurelien, Meng Wang, and Andres Ladino. 2019a. ``A Hierarchical
Approach for Splitting Truck Platoons Near Network Discontinuities,''
no. July: Submitted.

\leavevmode\hypertarget{ref-Duret2019}{}%
---------. 2019b. ``A Hierarchical Approach for Splitting Truck Platoons
Near Network Discontinuities.'' \emph{Submitted to 23rd International
Symposium on Transportation and Traffic Theory}.

\leavevmode\hypertarget{ref-Grandinetti2015}{}%
Grandinetti, Pietro, Carlos Canudas-de-wit, and Federica Garin. 2015.
``An efficient one-step-ahead optimal control for urban signalized
traffic networks based on an averaged Cell-Transmission Model.'' In
\emph{2015 European Control Conference (Ecc)}, 3478--83. Vienna,
Austria: IEEE. \url{https://doi.org/10.1109/ECC.2015.7331072}.

\leavevmode\hypertarget{ref-Grandinetti2016}{}%
Grandinetti, Pietro, Federica Garin, and Carlos Canudas-de-wit. 2015.
``Towards scalable optimal traffic control.'' In \emph{54th IEEE
Conference on Decision and Control (CDC 2015)}. Osaka, Japan.
\url{https://hal.archives-ouvertes.fr/hal-01188811}.

\leavevmode\hypertarget{ref-Ladino2018}{}%
Ladino, Andres, Carlos Canudas-de-wit, Alain Kibangou, Hassen Fourati,
and Martin Rodriguez. 2018. ``Density and flow reconstruction in urban
traffic networks using heterogeneous data sources.'' In \emph{2018
European Control Conference, Ecc 2018}, edited by IEEE. Limasol,
Chyprus.

\leavevmode\hypertarget{ref-Lovisari2016}{}%
Lovisari, E., Carlos Canudas-de-wit, and Alain Y Kibangou. 2016.
``Density/Flow reconstruction via heterogeneous sources and Optimal
Sensor Placement in road networks.'' \emph{Transportation Research Part
C: Emerging Technologies} 69: 451--76.
\url{https://doi.org/10.1016/j.trc.2016.06.019}.

\leavevmode\hypertarget{ref-Turri2017}{}%
Turri, Valerio, Bart Besselink, and Karl H. Johansson. 2017.
``Cooperative Look-Ahead Control for Fuel-Efficient and Safe Heavy-Duty
Vehicle Platooning.'' \emph{IEEE Transactions on Control Systems
Technology} 25 (1): 12--28.
\url{https://doi.org/10.1109/TCST.2016.2542044}.

\leavevmode\hypertarget{ref-Varaiya2013:TR-C}{}%
Varaiya, Pravin. 2013. ``Max pressure control of a network of signalized
intersections.'' \emph{Transportation Research Part C: Emerging
Technologies} 36 (2): 177--95.
\url{https://doi.org/10.1016/j.trc.2013.08.014}.

\leavevmode\hypertarget{ref-Meng2014b:TR-C}{}%
Wang, Meng, Winnie Daamen, Serge P. Hoogendoorn, and Bart van Arem.
2014. ``Rolling horizon control framework for driver assistance systems.
Part II: Cooperative sensing and cooperative control.''
\emph{Transportation Research Part C} 40: 290--311.
\url{https://doi.org/10.1016/j.trc.2013.11.024}.


\end{document}
